% Initialisation
\documentclass[english,12pt]{scrartcl}
\usepackage[]{babel}
% Input is utf8
\usepackage[utf8]{inputenc}
% Enables headers and footers
\usepackage[]{scrpage2}
% Lets us colour table cells
\usepackage[table]{xcolor}
% Allows todo list and todos
\usepackage[]{todonotes}
% Makes links in contents hyperlinked
\usepackage{hyperref}
% Make references appear in our table of contents
\usepackage[nottoc,numbib]{tocbibind}
% Allows us to put landscape sections of the document
\usepackage{pdflscape} % \usepackage{lscape} %Use escape for printing (doesn't rotate the pdf page)
% Provides a glossary
\usepackage[toc]{glossaries}

% Gives us pretty diagrams
\usepackage{tikz}
\usetikzlibrary{calc,fit,positioning,chains,decorations.pathreplacing,shapes,backgrounds}

% Document Title and Author
\title{Group Reflection}
\author{2013 Final Year Project}

% Header and Footer
\pagestyle{scrheadings}
\ihead{\today}
\chead{}
\ohead{Group Reflection}
\ifoot{}
\cfoot{}
\ofoot{\pagemark}

% Requirements custom commands
\newcommand{\requirement}[1]{\textit{#1}}

% Skip line rather then indent paragraphs
\setlength{\parindent}{0.0in}
\setlength{\parskip}{0.1in}

% Start of document
\begin{document}
	\maketitle
	\vfill
	{\large
		\begin{description}
			\item [Status:] Draft 1
			\item [Version:] 0.1
		\end{description}}

	\clearpage
	\tableofcontents
		
	\clearpage

\section{Project Quality}
	\subsection{NUClear}
		NUClear was the core part of the project, without this architectural framework our project would have been of no use to the NUBots team.
		NUClear was a huge success. NUClear has achieved its goals of being a fast, easy to use, flexible, loosely coupled framework.
		It has used innovative techniques such as template metaprogramming and compile time message routing to achieve this.
		It to allow the NUbots system to be much more expandable and flexible once NUClearPort is complete.
		It has also allowed for the easy construction of completely new systems such as Robot Dance and Mech Warrior using components already available from NUClearPort.
		It is also already being used for research purposes by the NUbots team.
	
	\subsection{NUClearPort}
		While NUClearPort did not progress as far as originally intended it is well on it's way to converting the old NUbots soccer system to NUClear.
		It has given the NUbots team a great starting point from which they intend to complete the transfer to the NUClear which they now be using.
		Rather than directly converting the old components to NUClear, our team was able to upgrade many components.
		NUClear has enabled new vital systems to be upgraded including networking, and debugging which will greatly improve further development.
		NUClear has enabled the NUbots system to make use of all processor cores on future robot hardware.
		The addition of roles has made it incredibly easy to various different system with little effort.
		Roles can be used for easily creating new unit tests for components or even to allow NUbots system components to be used for completely different purposes such as Robot Dance.
		While NUClearPort is not complete enough for the robot to play soccer, it is well on its way to doing so for the 2014 RoboCup, as well as adding many new useful features.
	
	\subsection{Robot Dance}
			Robot Dance has achieved its purpose of being able to dance in time to the beats of music.
			The core of the system is solid, though some components could use improvement.
			While the beat tracker works great for music with simple beats, it struggles with complicated music with more difficult beats.
			Additionally the audio input from microphone does not currently take into account noise from the robots fan and motors which is a major problem.
			Fortunately with the modular design of NUClear and the new roles system it will be easy to replace the current beat tracker with a better version.
			It will also be possible to add a filter component in between the audio input and beat tracking to take care of the problem of noise.
			Due to time constraints we were unable to implement these upgrades.

\section{Engineering Approaches}
	\subsection{Project Planning}
		While we were unable to complete all our estimated goals such as completing NUClearPort, we were able to be prepared for this.
		We focused on porting the core functionality for the NUbots system first, so that if we were unable to complete it they would have a much easier time finishing it.
		There were also some problems that occurred that we did not fully plan for as much as we could, such as having 2 team members leave the project.
		Because of this we were unable to do dancing in response to video.
		Fortunately we did not lose any of the team members working on NUClear as this would have caused major problems with the whole project since it all depends on NUClear.
	
	\subsection{Group Co-ordination}
		Our team met once a week for the past 2 semesters. This seems to have worked well for our team.
		Because our team was split into 2 distinct task groups more weekly meeting were not needed.
		The members working on each of the 2 tasks were able to manage their own communication.
		We also used BaseCamp for group communication particularly for discussing and sharing documentation.
		In the second semester when we went down to 4 members BaseCamp became less necessary as all the team members we able to communicate directly and git was used for sharing documents.
		Skype was also employed regularly for group communication.
	
	\subsection{Software Engineering Methodologies}
		
	\subsection{Documentation}
		Due to losing team members documentation became more of a burden than we originally estimated.
		Focus had to be allocated to finishing the documentation rather than making further progress with the project, particularly with NUClearPort.
		Many of our documents were delivered somewhat behind schedule, however they were still developed to a professional standard.
	
	\subsection{Tools}
		The tools we used for our project were Git, GitHub and BaseCamp.
		Many of our group used different IDE's and we made no effort to synchronise on what to use for this.
		Git and GitHub proved extremely useful.
		They allowed us to efficiently share code, to create different versions of the code to work on particular features and to easily go check previous code that we made.
		BaseCamp was useful particularly for first semester as it was easy to learn and more useful for a large team. It helped us setup other tools such as git and agree on our approach when the project was new.
		It however became less useful in semester 2 as the group became smaller and we had already determined our approaches.
\section{Overall Development}
	%TODO More stuff in this section?
	\subsection{Things to improve}
		The work on NUClearPort while excellent, is still incomplete.
		It was desired that we could have completed this however we were unable due to team members leaving and the need to complete the documentation.
		This however will be completed by the NUbots team.
		Robot dance, while able to fulfil its objectives, has much room for improvement.
		The beat tracking algorithm could be improved to use a more modern beat tracker.
		Additionally a noise filter could be implemented to remove background noise, particularly fan and motor noise from the robot, from the audio signal.
	
	\subsection{What the group has learned}
	
\end{document}
