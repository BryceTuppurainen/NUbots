% Initialisation
\documentclass[english,12pt]{scrartcl}
\usepackage[]{babel}
% Input is utf8
\usepackage[utf8]{inputenc}
% Enables headers and footers
\usepackage[]{scrpage2}
% Lets us colour table cells
\usepackage[table]{xcolor}
% Allows todo list and todos
\usepackage[]{todonotes}
% Makes links in contents hyperlinked
\usepackage{hyperref}
% Make references appear in our table of contents
\usepackage[nottoc,numbib]{tocbibind}
% Allows us to put landscape sections of the document
\usepackage{pdflscape} % \usepackage{lscape} %Use escape for printing (doesn't rotate the pdf page)
% Provides a glossary
\usepackage[toc]{glossaries}

% Gives us pretty diagrams
\usepackage{tikz}
\usetikzlibrary{calc,fit,positioning,chains,decorations.pathreplacing,shapes,backgrounds}

% Document Title and Author
\title{Group Reflection}
\author{2013 Final Year Project}

% Header and Footer
\pagestyle{scrheadings}
\ihead{\today}
\chead{}
\ohead{Group Reflection}
\ifoot{}
\cfoot{}
\ofoot{\pagemark}

% Requirements custom commands
\newcommand{\requirement}[1]{\textit{#1}}

% Skip line rather then indent paragraphs
\setlength{\parindent}{0.0in}
\setlength{\parskip}{0.1in}

% Start of document
\begin{document}
	\maketitle
	\vfill
	{\large
		\begin{description}
			\item [Status:] Draft 1
			\item [Version:] 0.1
		\end{description}}

	\clearpage
	\tableofcontents
		
	\clearpage

\section{Project Quality}
	\subsection{NUClear}
		NUClear was the core part of the project, without this architectural framework our project would have been of no use to the NUBots team.
		NUClear was a huge success. NUClear has achieved its goals of being a fast, easy to use, flexible, loosely coupled framework.
		It has used innovative techniques such as template metaprogramming and compile time message routing to achieve this.
		It to allow the NUbots system to be much more expandable, maintainable and flexible once NUClearPort is complete.
		NUClear's message based design makes it easy to extend the the system easily plugging in new components or replacing old components.		
		In particular it's ability to allow the NUbots to move to a new robot platform as the hardware IO is no longer tightly coupled to the rest of the system.
		It has also allowed for the easy construction of completely new systems such as Robot Dance and Mech Warrior using components already available from NUClearPort.
		NUClear was designed with debugability as a core goal, making the system much more maintainable.
		NUClear has been embraced by the NUBots team who are continuing the work on NUClearPort.
		Additionally the NUbots team are already using NUClear to assist in robot research.
	
	\subsection{NUClearPort}
		While NUClearPort did not progress as far as originally intended it is well on it's way to converting the old NUbots soccer system to NUClear.
		It has given the NUbots team a great starting point from which they intend to complete the transfer to the NUClear which they now be using.
		Rather than directly converting the old components to NUClear, our team was able to upgrade many components.
		NUClear has enabled new vital systems to be upgraded including networking, and debugging which will greatly improve further development.
		NUClear has enabled the NUbots system to make use of all processor cores on future robot hardware.
		The addition of roles has made it incredibly easy to use the same components for various different system with little effort.
		Roles can be used for easily creating new unit tests for components or even to allow NUbots system components to be used for completely different purposes such as Robot Dance.
		Each component of the system is designed to be cohesive and the framework ensures that the components are very loosely coupled.
		This greatly improves the ability to maintain the system, as new members only need to learn how the single component they are trying to fix or improve works.
		This maintainability will be further improved when NUbugger is upgraded to take on the improvements to debugging built into NUClear.
		While NUClearPort is not complete enough for the robot to play soccer, it is well on its way to doing so for the 2014 RoboCup, as well as adding many new useful features.
	
	\subsection{Robot Dance}
			Robot Dance has achieved its purpose of being able to dance in time to the beats of music.
			The core of the system is solid, though some components could use improvement.
			While the beat tracker works great for music with simple beats, it struggles with complicated music with more difficult beats.
			Additionally the audio input from microphone does not currently take into account noise from the robots fan and motors which is a major problem.
			Fortunately with the modular design of NUClear and the new roles system it will be easy to replace the current beat tracker with a better version.
			It will also be possible to add a filter component in between the audio input and beat tracking to take care of the problem of noise.
			Due to time constraints we were unable to implement these upgrades.
			The dance system excels however in it's ability to easily add more dance scripts. 
			We have developed a seperate role that allows the user to create dance moves by moving the robot into a series of positions and taking a snapshot of those positions. 
			A script will then be created which causes the robot to move between these positions.
			 I.e. Do a dance move.

\section{Engineering Approaches}
	\subsection{Project Planning}
		While we were unable to complete all our estimated goals such as completing NUClearPort, we were able to be prepared for this.
		We focused on porting the core functionality for the NUbots system first, so that if we were unable to complete it they would have a much easier time finishing it.
		There were also some problems that occurred that we did not fully plan for as much as we could, such as having 2 team members leave the project.
		Because of this we were unable to do dancing in response to video.
		Fortunately we did not lose any of the team members working on NUClear as this would have caused major problems with the whole project since it all depends on NUClear.
	
	\subsection{Group Co-ordination}
		Our team met once a week for the past 2 semesters. This seems to have worked well for our team.
		Because our team was split into 2 distinct task groups more weekly meeting were not needed.
		The members working on each of the 2 tasks were able to manage their own communication.
		We also used BaseCamp for group communication particularly for discussing and sharing documentation.
		In the second semester when we went down to 4 members BaseCamp became less necessary as all the team members we able to communicate directly and git was used for sharing documents.
		Skype was also employed regularly for group communication.
	
	\subsection{Software Engineering Methodologies}
		Due to our close communication with the NUbots team as well as our small team size, our group approach evolved into a semi-agile approach. This was expressed through the use of an evolutionary prototype which was used in the design and development of NUClear. It allowed us to see what could be done with the tools we had available. We were however constrained in following agile development to some extent by the quantity of documentation required for the project.
		
	\subsection{Documentation}
		Due to losing team members documentation became more of a burden than we originally estimated.
		Focus had to be allocated to finishing the documentation rather than making further progress with the project, particularly with NUClearPort.
		Many of our documents were delivered somewhat behind schedule, however they were still developed to a professional standard.
		For our project we developed the following documents:
		\begin{itemize}
			\item Project Plan
			\item Requirement Document
			\item Design Document
			\item Test Plan
			\item Group Reflection (This Document)
			\item NUbots training document
		\end{itemize}
		
		Due to our agile approach and close communication with the NUbots team, at times the documentation seemed more of a burden than actually useful. 
		The project plan however helped us to consider the risks that could take place in our project.
		Also the requirements document helped us to fully clarify some of the requirements we had previously overlooked.
		The test plan was useful for the NUbots team to use in the future to ensure future improvements are correct.
		By far the most useful of the documents was the NUbots training document.
		This document was developed to teach current and future NUbots members how to use NUClear and NUClearPort.
		This document will helping NUBots members to use the new architecture we developed, to continue to port the system to NUClear and for all future NUbots development.

		
	
	\subsection{Tools}
		The tools we used for our project were Git, GitHub, BaseCamp, Latex and C++.
		Many of our group used different IDE's and we made no effort to synchronise on what to use for this.
		\\
		Git and GitHub proved extremely useful.
		They allowed us to efficiently share code, to create different versions of the code to work on particular features and to easily go check previous code that we made. They were also incredibly useful for documentation control.
		\\
		BaseCamp was useful particularly for first semester as it was easy to learn and more useful for a large team. 
		It allowed us to have a common discussion board for the issues in our project. It was also useful for sharing files.
		It helped us setup other tools such as git and agree on our approach when the project was new.
		It however became less useful in semester 2 as the group became smaller when we had already determined our approaches and found it easier to communicate directly using other methods.
		\\
		We used Latex for writing our documentation. 
		Latex turned out to be an excellent choice as a document writer. 
		It allowed us to effectively share our documents using git.
		It is also a flexible tool, allowing many things to be done with a document.
		Although several of our team had no prior Latex experience, we quickly acquired a knack for using it.
		\\
		C++ was the programming language we used for our project. 
		It was required that we use C++ as that is what the NUbots team uses. 
		C and C++ are the standard languages used for robotics. 
		C++ is able to work on a low level, making it excellent for communication with hardware such as the Darwin robots used by the NUbots.
		C++ is also an object oriented language which makes it suitable for the large complicated system we were working on.
		Our team was able to make full use of C++ using innovative techniques such as template meta-programming to make NUClear quick and very easy to use.
		We managed to make use of nearly all the features released in the latest version of C++ (C++11).
		The downside to this is that while NUClear is incredible easy to use for the user, it is very difficult to modify the core code of NUClear unless you are experienced with template meta-programming.
		Some of our group members had little C++ experience at the start of the project.
		Those group members had their skills greatly improved through the course of the project.
		Nonetheless, this lack of experience at the start of the project caused some setbacks, particularly for the robot dance.
		
\section{Overall Development}
	%TODO More stuff in this section?
	\subsection{Things to improve}
		The work on NUClearPort while excellent, is still incomplete.
		It was desired that we could have completed this however we were unable due to team members leaving and the need to complete the documentation.
		This however will be completed by the NUbots team.
		Robot dance, while able to fulfil its objectives, has much room for improvement.
		The beat tracking algorithm could be improved to use a more modern beat tracker.
		Additionally a noise filter could be implemented to remove background noise, particularly fan and motor noise from the robot, from the audio signal.
	
	\subsection{What the group has learned}
	
\end{document}
