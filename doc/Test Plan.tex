\documentclass[a4paper]{article}

\title{Title}
\author{Jake Woods, Trent Houliston}
\date{\today}

\usepackage{tikz}


\begin{document}

\maketitle

\section{Introduction}
	This document outlines the scope of testing applied to the implementation of the new architecture for the robocup codebase. 
	The original architecture relied on purely manual processes for all testing. 
	We believe that the new architecture should utilize automated testing whenever feasible while using manual testing as a last-resort backup.
	
	In the new architecture functionality is seperate into discrete modules that communicate through messages. 
	Accordingly we can seperate our unit tests into per-module tests providing clean seperation of concerns and simple mechanisms for mocking and validating inputs/outputs.
	We have also identitfied a number of cases where integration tests can be performed on module subsets further reducing the complexity of testing the system.
	End-to-end testing will still require manual validation as fully automated end-to-end testing would require an emulation layer for the robots hardware that was outside the scope of this project.

\section{Scope}
	The following modules can be automatically unit tested:
	\begin{itemize}
		\item AubioBeatDetector
		\item BeatDetector
		\item ConfigSystem
		\item PartyDarwin
		\item ScriptEngine
		\item ScriptRunner
	\end{itemize}

	The following module combinations can be combined into automatic integration tests:
	\begin{itemize}
		\item AudioFileInput, (AubioBeatDetector or BeatDetector)
	\end{itemize}

	These modules require manual testing because they depend on robot hardware that we can't emulate:
	\begin{itemize}
		\item AudioInput
		\item eSpeak
		\item LinuxCameraStreamer
		\item NUBugger
		\item Platform/Darwin/HardwareIO
		\item Platform/Darwin/MotionManager
	\end{itemize}
	
	
\section{Modules}
	\subsection{AubioBeatDetector}
		\subsubsection{Description}
			AubioBeatDetector detects beats in a supplied audio stream using the Aubio library.
			Beats are reported as the time the beat occured and the tempo.
		\subsubsection{Testing Method}
			AubioBeatDetector can be automatically unit tested using the AudioFileInput reactor to supply mock sound from a known sound file. 
			We can then check that the detected beat is "close enough" to the known value for the supplied data.
		\subsubsection{Inputs}
			AubioBeatDetector expected to be initialized with a SoundChunkSettings message that informs it of information relating to the input device.
			It also expects a periodic SoundChunk message containing the data that was recorded.
		\subsubsection{Outputs}
			AubioBeatDetector should emit a Beat message every time it detects a beat in the sound sample.

	\subsection{AudioFileInput}
		\subsubsection{Description}
			AudioFileInput reads from a wav file specified in it's configuration. 
			It the emits SoundChunk messages periodically as if it was listening to the file through the input device.
		\subsubsection{Testing Method}
			Can't test, fill in details later. Consider unchunking it's output to see if it forms the same file? Not worth it probably.

	\subsection{AudioInput}
		\subsubsection{Description}
			AudioInput reads from the default audio input device.
			It emits SoundChunk messages periodically as more data is read in from the input device.
		\subsubsection{Testing Method}
			Can't test, fill in details later.

	\subsection{BeatDetector}
		\subsubsection{Description}
			BeatDetector detects beats in a supplied audio stream using the a custom algorithm [FILL IN DETAILS].
			Beats are reported as the time the beat occured and the tempo.
		\subsubsection{Testing Method}
			BeatDetector can be automatically unit tested using the AudioFileInput reactor to supply mock sound from a known sound file. 
			We can then check that the detected beat is "close enough" to the known value for the supplied data.
		\subsubsection{Inputs}
			BeatDetector expected to be initialized with a SoundChunkSettings message that informs it of information relating to the input device.
			It also expects a periodic SoundChunk message containing the data that was recorded.
		\subsubsection{Outputs}
			BeatDetector should emit a Beat message every time it detects a beat in the sound sample.

	\subsection{ConfigSystem}
		\subsubsection{Description}
			ConfigSystem loads JSON-formatted configuration files and supplies them to other modules.
		\subsubsection{Testing Method}
			
		
\end{document}