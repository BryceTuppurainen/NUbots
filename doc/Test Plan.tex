\documentclass[a4paper]{article}

\title{Title}
\author{Jake Woods, Trent Houliston}
\date{\today}

\usepackage{tikz}


\begin{document}

\maketitle

\section{Introduction}
	This document outlines the scope of testing applied to the implementation of the new architecture for the robocup codebase. 
	The original architecture relied on purely manual processes for all testing. 
	We believe that the new architecture should utilize automated testing whenever feasible while using manual testing as a last-resort backup.
	
	In the new architecture functionality is seperate into discrete modules that communicate through messages. 
	Accordingly we can seperate our unit tests into per-module tests providing clean seperation of concerns and simple mechanisms for mocking and validating inputs/outputs.
	We have also identitfied a number of cases where integration tests can be performed on module subsets further reducing the complexity of testing the system.
	End-to-end testing will still require manual validation as fully automated end-to-end testing would require an emulation layer for the robots hardware that was outside the scope of this project.

\section{Scope}
	The following modules can be automatically unit tested:
	\begin{itemize}
		\item AubioBeatDetector
		\item BeatDetector
		\item ConfigSystem
		\item PartyDarwin
		\item ScriptEngine
		\item ScriptRunner
	\end{itemize}

	The following module combinations can be combined into automatic integration tests:
	\begin{itemize}
		\item AudioFileInput, (AubioBeatDetector or BeatDetector)
	\end{itemize}

	These modules require manual testing because they depend on robot hardware that we can't emulate:
	\begin{itemize}
		\item AudioInput
		\item eSpeak
		\item LinuxCameraStreamer
		\item NUBugger
		\item Platform/Darwin/HardwareIO
		\item Platform/Darwin/MotionManager
	\end{itemize}
	
	
\section{Modules}
	\subsection{AubioBeatDetector}
		\subsubsection{Description}
			The AubioBeatDetector detects beats in a supplied audio stream.
			Beats are reported as the time the beat occured and the tempo.
		\subsubsection{Inputs}

\end{document}